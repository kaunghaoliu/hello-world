\documentclass[12pt]{article}%scrartcl

%\usepackage[nodisplayskipstretch]{setspace}
\usepackage{url}
\usepackage{amssymb, amsmath}
\usepackage{graphicx}
\usepackage[tight,footnotesize]{subfigure}
\usepackage{times}
\usepackage{caption}
\usepackage{enumitem} % Adjust item space in a list
\usepackage{multirow}
\usepackage{hyperref}
\usepackage{mathtools}
%\renewcommand*\rmdefault{ppl}\normalfont\upshape
%-----------------------------------------------------------
%Margin setup
\usepackage
[
        %a4paper,% other options: a3paper, a5paper, etc
        left=1.2cm,
        right=1.2cm,
        top=2cm,
        bottom=2cm,
        % use vmargin=2cm to make vertical margins equal to 2cm.
        % us  hmargin=3cm to make horizontal margins equal to 3cm.
        % use margin=3cm to make all margins  equal to 3cm.
]
{geometry}
%-----------------------------------------------------------
%%% Headers and footers
\usepackage{fancyhdr}                                    % Needed to define custom headers/footers
    \pagestyle{fancy}

    %\fancypagestyle{plain}{ %
      %\fancyhf{} % remove everything
      \rhead[]{Draft}
      \lhead[IRI]{IRI}
%}

% Customize the first page
\fancypagestyle{plain}{%
  \renewcommand{\headrulewidth}{1pt}%
  \rhead[]{Spring 2016}
  \lhead[]{Senior Project, NCKU}
  %\fancyfoot[C]{\footnotesize Page \thepage\ of \pageref{LastPage}}%
}
%-----------------------------------------------------------

%%\def \myMode {}
%\ifx \myMode \undefined
%% --- Chinese in Windows (\my mode is empty) ---
%    \usepackage{fontspec}                  %加這個就可以設定字體
%    \usepackage[BoldFont]{xeCJK}           %讓中英文字體分開設置
%    \setCJKmainfont{標楷體}                 %設定中文為系統上的字型,而英文不變動,使用原TeX字型
%    \XeTeXlinebreaklocale "zh"             %這兩行一定要加,中文才能自動換行
%    \XeTeXlinebreakskip = 0pt plus 1pt     %這兩行一定要加,中文才能自動換行
%    \setlength{\parskip}{0.3cm} %設定段落之間的距離
%    \linespread{1.5}\selectfont %設定行距
%    %\renewcommand{\baselinestretch}{1.5}
%\else
%    \usepackage{fontspec}
%    \usepackage[BoldFont]{xeCJK}           %讓中英文字體分開設置
%    \setromanfont[AutoFakeBold=6]{STKaiti} %楷體, Yosemine
%    %\setromanfont{STFangsong} % 仿宋
%    %\setromanfont{STSong} % 宋體
%    %\setromanfont{PMingLiU}% 新細明體
%    %\setromanfont{STHeiti} %黑體
%    %\setromanfont[AutoFakeBold=6]{BiauKai} %舊版
%    %\setromanfont{Apple LiGothic}
%    % option: Apple LiSung Light
%    \XeTeXlinebreaklocale "zh"             %這兩行一定要加,中文才能自動換行
%    \XeTeXlinebreakskip = 0pt plus 1pt     %這兩行一定要加,中文才能自動換行
%    \setlength{\parskip}{0.3cm} %設定段落之間的距離
%    \linespread{1.2}\selectfont %設定行距
%\fi
%-----------------------------------------------------------

\begin{document}
\title{Wireless Power Transfer for Cooperative Communication Systems}
\date{}
\author{Kuang-Hao (Stanley) Liu}
%\author{}
\maketitle

This article is prepared as an instruction for students who need to
conduct their senior project in the area of wireless communications.
Among many interesting topics, this article focuses on the one shown
in the title. Through working closely with me, students will learn
how to do research step by step and how to present the research
results. Enjoy your journey and the joy when reaching the goal.

\paragraph{Copyright Notice}
Copyright \copyright 2016 by Kuang-Hao Liu. All rights reserved. No
part of this document may be reproduced, distributed, or transmitted
in any form or by any means, including photocopying, recording, or
other electronic or mechanical methods, without the prior written
permission of the author. For permission requests, write to the
author by email at khliu@mail.ncku.edu.tw.


\section{Background}
The global energy crisis has urged the development of green
technologies that aim to improve energy utilization. Energy
harvesting (EH) is one of the green technologies that extracts
energy from ambient sources (e.g. solar, wind, thermal energy,
vibrations, electromagnetic energy, radio frequency (RF) signals,
and so on) and converts the collected energy into usable electrical
energy for the operation of devices~\cite{Vullers2010}. In this
work, we focus on RF-based EH, namely, wireless power transfer
(WPT), which reuses the energy for transmitting information to
receive and decode information. Various applications (see
Fig.~\ref{fig:example}) can benefit from RF-based EH such as
wireless sensor networks~\cite{Sudevalayam2011} and Internet of
Things (IoT) where trillions of communications devices can be
installed without power-grid planning or frequent battery
replacement.

\begin{figure}[ht]
\centering
\includegraphics[width=0.5\columnwidth]{example.eps}
\caption{Example applications of wireless power
transfer~\cite{Bi2015}.} \label{fig:example}
\end{figure}

Based on the idea of energy reuse, EH has been recently applied to
cooperative communications~\cite{Liu2015}, where the information
delivery from a source node to a destination node is assisted via
one or several intermediate nodes, referred to as relay nodes. These
relay nodes perform information relaying (IR) by using the energy
harvested from the source signal and thus do not consume energy from
external sources. In addition to enhanced transmission coverage and
reliability, IR with multiple relay nodes can benefit from spatial
diversity provided with sophisticated space-time codes. To ease the
implementation complexity, a viable solution is to select a single
out of multiple relay nodes for IR.

\section{Project Goal}
Traditionally, relay nodes are powered by fixed energy supplies that
need to wire power cords or replace batteries frequently. To
mitigate troublesome wiring and battery replacement, recent advances
on energy harvesting techniques have shown the great potential of
charing devices using renewable energy. Among various renewable
energy sources, charging via the RF signal is of particular
interesting the RF signal inherently carries both information and
energy. The carried energy can be readily collected because of the
broadcast nature of wireless transmissions.

This project will be devoted to address challenging issues when
applying RF-based energy harvesting to relay nodes. Since the
harvested energy from RF signal is often small and it incurs a
certain loss during the conversion process from RF to DC, it is of
paramount importance to use the harvested energy efficiently. In
this project, students will investigate how to embrace the benefits
of energy harvesting and cooperative communications considering the
characteristics and limitations of both.

\section{Research Steps}

\paragraph{Stage I: Broad View}% (蒐集相關文獻) 2016年3月}
\begin{enumerate}
\item Understand basic principles of RF energy harvesting (e.g, how the energy is collected and stored, what
are the physical constraints, etc.)
 \begin{itemize}
 \item Read the article~\cite{Bi2015} from p. 117 to p.~119, not
 including the section ``Energy Beamforming''.
 \end{itemize}
\item Understand the receiver structure for wireless power transfer
(WPF).
 \begin{itemize}
  \item Read the article~\cite{Krikidis2014}: Section I to Section
  IV.
 \end{itemize}
%what are the critical and challenging issues of applying
%RF energy harvesting in wireless communication systems (e.g.,
%dynamics of charging and discharging processes; unstable wireless
%channels due to multipath fading, shadowing, and pathloss;
%inaccurate information about the system status,etc.). 3. Identify
%the specific challenges of different applications e.g., sensor
%networks, cellular networks, cooperative networks, cognitive
%networks, etc.
\item Understand how EH and WPT are applied to cooperative
communication systems.
 \begin{itemize}
 \item Read the article~\cite{Liu2015}: page 56-60.
 \end{itemize}
\end{enumerate}

%In~\cite{Kung2015}, a new relay selection scheme is proposed for EH
%relays together with the idea of time reuse for throughput
%improvement. Some potential extensions to~\cite{Kung2015} include
%
%\begin{itemize}
%\item Fade duration: Fast fading (block fading) is assumed in~\cite{Kung2015}. In
%practice, the channel is more likely static across several blocks.
%\item Energy storage: How to utilize limited storage?
%\item EH during relay transmission: How to make use of relay signal
%in additional to source signal for EH?
%\item Beamforming: Adjust beams to maximize EH efficiency.
%\end{itemize}



\paragraph{Stage II: Simulation}% (建立模擬平台)}
Further understand a cooperative communication system with EH
relays.
\begin{itemize}
\item Read Sections II and III in~\cite{Nasir2013} and the simulation code (download from \href{https://drive.google.com/open?id=0B-ZxWv0dsZGAOGFEQ0FMWktzY28}{here}).
 \begin{itemize}
  \item The sample code considers one source node, one destination node, and four relays. The source transmits the same signal to all the relays and each relay forwards the received signal to the destination. Refer to for the system model.
  \item The channel used to transmit the signal is assumed to experience {\it Rayleigh fading}. See~\cite{RayleighFadingChannel} for a good tutorial for Rayleigh fading channel.
  \item Notice that in the sample code, the end-to-end SNR differs from the expression of Eq.~(10) in~\cite{Nasir2013}. You will need to modify the code accordingly.
  \item Modify the sample code to simulate the outage probability. Verify the simulation results with Eq.~(12a) in~\cite{Nasir2013}.
  \item Show your simulation code and results in the next project
  meeting.
    \end{itemize}
\end{itemize}

\paragraph{Stage III: Problem Formulation}% (定義研究問題)}
Identify the research problem of your interest.
\begin{itemize}
\item You should have enough background on cooperative communication
systems with EH relays. Discuss with me to learn some of the open
issues and identify the one to be your own research topic.
\end{itemize}

\paragraph{Stage IV: Solution Development}% (發展問題解決方法)}
Once your research topic is defined, it is time to develop the
solution. At this stage, you are encouraged to start with writing
the simulation code for the system under consideration.

\paragraph{Stage V: Verification and Interpretation} %(驗證並分析結果)}
Now you should have some results obtained from running your
simulation code. It is important to verify the correctness of your
results and interpret them. Again, make an appointment with me for
discussion.

\paragraph{State VI: Wrap Up}% (整理研究成果為專題報告)}
Conclude your project by putting everything together in the form of
a project report. You will be provided with a template and a sample
for preparing your report.

%\paragraph{Stage III: Problem formulation}
%
%TBD.
%
%\paragraph{Stage IV: Solution development}
%
%TBD.
%
%\paragraph{Stage V: Results validation}
%
%TBD.
%
%\paragraph{Stage VI: Writing thesis}
%
%TBD.

%\begin{thebibliography}{10}
%
%\bibitem{Adachi2012}
%K.~Adachi, J.~Joung, S.~Sun, and P.~H.~Tan, ``Energy-saving coordinated napping
%(CoNap) for wireless networks,'' in \emph{Proc. IEEE Globecom'12}, Anaheim, CA,
%Dec. 3-7, 2012.
%
%\end{thebibliography}

%\section{Reference}
%{\footnotesize
\bibliographystyle{IEEEtran}
\bibliography{madmf}
%}







\end{document}
